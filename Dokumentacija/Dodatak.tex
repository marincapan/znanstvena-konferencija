\chapter*{Dodatak: Prikaz aktivnosti grupe}
		\addcontentsline{toc}{chapter}{Dodatak: Prikaz aktivnosti grupe}
		
		\section*{Dnevnik sastajanja}
		
		\textbf{\textit{Kontinuirano osvježavanje}}\\
		
		 \textit{U ovom dijelu potrebno je redovito osvježavati dnevnik sastajanja prema predlošku.}
		
		\begin{packed_enum}
			\item  sastanak
			
			\item[] \begin{packed_item}
				\item Datum: 11. listopada 2021.
				\item Prisustvovali: J.Gavran, L.Mujagić, M.Capan, P.D.Grujić Ostojić, R.Pintar, F.Mesić, A.Sambolek 
				\item Teme sastanka:
				\begin{packed_item}
					\item Na prvom smo se neformalnom sastanku upoznali i kratko prodiskutirali dodijeljeni projektni zadatak složivši se da nećemo predlagati vlastiti projektni zadatak.
					
					\item Okvirno smo dogovorili podjelu uloga unutar tima koju ćemo po potrebi revidirati. Članovi tima će si međusobno pomagati te će svatko imati priliku (i obvezu)
					raditi i na implementaciji i na dokumentaciji projekta. Po trenutnoj su podjeli uloga za poslužiteljsku stranu aplikacije zaduženi M.Capan i F.Mesić, za klijentsku
					stranu R.Pintar i L.Mujagić, za dokumentaciju zahtjeva A.Sambolek, za bazu podataka P.D.Grujić Ostojić te za dizajn korisničkog sučelja/korisničkog iskustva J.Gavran.
					
					\item Razgovarali smo koje bismo tehnologije i alate koristili pri razvoju aplikacije, a koje za komunikaciju unutar tima. Inicijalni je dogovor da ćemo koristiti Python,
					Bootstrap i PostgreSQL, a komunicirat ćemo preko WhatsAppa i Discorda.
				\end{packed_item}
			\end{packed_item}
			
			\item  sastanak
			\item[] \begin{packed_item}
				\item Datum: 13. listopada 2021.
				\item Prisustvovali: J.Gavran, L.Mujagić, M.Capan, P.D.Grujić Ostojić, R.Pintar, F.Mesić, A.Sambolek
				\item Teme sastanka:
				\begin{packed_item}
					\item  Na ovom inicijalnom sastanku s asistentom Miljenkom Krhenom bili su prisutni svi članovi tima i demonstrator zadužen za našu grupu, kolega Vedran Kolka.
					Asistent nam je dao osnovne informacije o načinu provedbe i kolokviranju projekta. Pozvani smo dolaziti na fakultet u terminima laboratorijskih vježbi kako bismo
					diskutirali projektno rješenje i nedoumice, a također pitanja možemo postavljati i preko platforme MS Teams.
					
				\end{packed_item}
			\end{packed_item}
			\item  sastanak
		\item[] \begin{packed_item}
			\item Datum: 
			\item Prisustvovali:
			\item Teme sastanka:
			\begin{packed_item}
				\item
			\end{packed_item}
		\end{packed_item}
			
			%
			
		\end{packed_enum}
		
		\eject
		\section*{Tablica aktivnosti}
		
			\textbf{\textit{Kontinuirano osvježavanje}}\\
			
			 \textit{Napomena: Doprinose u aktivnostima treba navesti u satima po članovima grupe po aktivnosti.}

			\begin{longtblr}[
					label=none,
				]{
					vlines,hlines,
					width = \textwidth,
					colspec={X[7, l]X[1, c]X[1, c]X[1, c]X[1, c]X[1, c]X[1, c]X[1, c]}, 
					vline{1} = {1}{text=\clap{}},
					hline{1} = {1}{text=\clap{}},
					rowhead = 1,
				} 
				\multicolumn{1}{c|}{} & \multicolumn{1}{c|}{\rotatebox{90}{\textbf{Marin Capan}}} & \multicolumn{1}{c|}{\rotatebox{90}{\textbf{Luka Mujagić }}} &	\multicolumn{1}{c|}{\rotatebox{90}{\textbf{Petra Dunja Grujić Ostojić }}} & \multicolumn{1}{c|}{\rotatebox{90}{\textbf{Fran Mesić}}} &	\multicolumn{1}{c|}{\rotatebox{90}{\textbf{Rea Pintar}}} & \multicolumn{1}{c|}{\rotatebox{90}{\textbf{Antonio Sambolek}}} &	\multicolumn{1}{c|}{\rotatebox{90}{\textbf{Jelena Gavran }}} \\  
				Upravljanje projektom 		&  &  &  &  &  &  & \\ 
				Opis projektnog zadatka 	&  &  &  &  &  &  & \\ 
				
				Funkcionalni zahtjevi       &  &  &  &  &  &  &  \\ 
				Opis pojedinih obrazaca 	&  &  &  &  &  &  &  \\ 
				Dijagram obrazaca 			&  &  &  &  &  &  &  \\ 
				Sekvencijski dijagrami 		&  &  &  &  &  &  &  \\ 
				Opis ostalih zahtjeva 		&  &  &  &  &  &  &  \\ 

				Arhitektura i dizajn sustava	 &  &  &  &  &  &  &  \\ 
				Baza podataka				&  &  &  &  &  &  &   \\ 
				Dijagram razreda 			&  &  &  &  &  &  &   \\ 
				Dijagram stanja				&  &  &  &  &  &  &  \\ 
				Dijagram aktivnosti 		&  &  &  &  &  &  &  \\ 
				Dijagram komponenti			&  &  &  &  &  &  &  \\ 
				Korištene tehnologije i alati 		&  &  &  &  &  &  &  \\ 
				Ispitivanje programskog rješenja 	&  &  &  &  &  &  &  \\ 
				Dijagram razmještaja			&  &  &  &  &  &  &  \\ 
				Upute za puštanje u pogon 		&  &  &  &  &  &  &  \\  
				Dnevnik sastajanja 			&  &  &  &  &  &  &  \\ 
				Zaključak i budući rad 		&  &  &  &  &  &  &  \\  
				Popis literature 			&  &  &  &  &  &  &  \\  
				&  &  &  &  &  &  &  \\ \hline 
				\textit{Dodatne stavke kako ste podijelili izradu aplikacije} 			&  &  &  &  &  &  &  \\ 
				\textit{npr. izrada početne stranice} 				&  &  &  &  &  &  &  \\  
				\textit{izrada baze podataka} 		 			&  &  &  &  &  &  & \\  
				\textit{spajanje s bazom podataka} 							&  &  &  &  &  &  &  \\ 
				\textit{back end} 							&  &  &  &  &  &  &  \\  
				 							&  &  &  &  &  &  &\\ 
			\end{longtblr}
					
					
		\eject
		\section*{Dijagrami pregleda promjena}
		
		\textbf{\textit{dio 2. revizije}}\\
		
		\textit{Prenijeti dijagram pregleda promjena nad datotekama projekta. Potrebno je na kraju projekta generirane grafove s gitlaba prenijeti u ovo poglavlje dokumentacije. Dijagrami za vlastiti projekt se mogu preuzeti s gitlab.com stranice, u izborniku Repository, pritiskom na stavku Contributors.}
		
	