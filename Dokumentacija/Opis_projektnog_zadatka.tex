\chapter{Opis projektnog zadatka}
		
		\textbf{\textit{dio 1. revizije}}\\

		Znanstvena udruga “Pametna ekipa” organizira konferenciju na kojoj će sudionici prezentirati svoje radove. Cilj ovog projekta je razviti učinkovit informacijski sustav koji će omogućiti sudionicima i recenzentima da obavljaju svoju funkciju (predaja i prezentacija te recenziranje radova), a administratoru upravljanje upitnicima i sadržajem na samoj web aplikaciji. Dodatno, web aplikacija sadrži informacije o trenutnoj epidemiološkoj situaciji te ukoliko se konferencija bude morala držati online svi sudionici će biti obaviješteni putem mail-a te će isto pisati na stranici.

		Prilikom pokretanja, neregistriranom, odnosno neprijavljenom korisniku prikazuje se naslovnica na kojoj se nalazi statičan sadržaj (informacije o konferenciji, nekoliko članaka o konferenciji/radovima i sl.) te gumb koji vodi do stranice za registraciju/prijavu.

		Na stranici za prijavu već registrirani korisnici prijavljuju se mail-om i lozinkom, dok korisnici koji nisu izvršili registraciju imaju dvije opcije: Registriraj se kao sudionik i Registriraj se kao recenzent. Administrator(i) i predsjedavajući konferencije nisu potrebni vršiti registraciju, kako ima jedan ustanovljen („hardkodiran“) administrator koji zatim dodjeljuje uloge drugima (imenuje predsjedavajućeg konferencije te ako želi – još jednog administratora).

		Za kreiranje korisničkog računa potrebni su sljedeći podaci:

		\begin{packed_item}

			\item ime i prezime
			\item korisnička oznaka
			\item email adresa
			\item matična ustanova i adresa iste - ulica i kućni broj, grad i država
			\item sekcija na kojoj žele sudjelovati/recenzirati radove
			\item potrebno je odabrati prijavljujen li se osoba kao sudionik, recenzent ili oboje
			\item imena autora rada (samo za sudionike)
		
		\end{packed_item}

		Nakon izvršene registracije, ovisno o tome koju ulogu je korisnik odabrao (sudionik, recenzent ili oboje) dobit će određena prava koja su objašnjena u nastavku.
		
		\eject
		
	