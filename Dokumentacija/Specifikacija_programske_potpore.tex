\chapter{Specifikacija programske potpore}
		
	\section{Funkcionalni zahtjevi}
			
			\textbf{\textit{dio 1. revizije}}\\
			
			
			\noindent \textbf{Dionici:}
			
			\begin{packed_enum}
				
				\item Sudionici konferencije
				\item Recenzenti
				\item Predsjedavajući konferencije				
				\item Administrator sustava
				
			\end{packed_enum}
			
			\noindent \textbf{Aktori i njihovi funkcionalni zahtjevi:}
			
			
			\begin{packed_enum}
				\item  \underbar{Neregistriran/neprijavljen korisnik (inicijator) može:}
				
				\begin{packed_enum}
					
					\item pristupiti naslovnici konferencije i sadržaju na njoj, kao i stranici s informacijama o samoj konferenciji
					\item prijaviti se u sustav, a ako to još nije napravio može se registrirati - stvoriti novi korisnički račun za koji su mu potrebni osobni podaci - ime i prezime, naziv matične ustanove (kao i ulica i kućni broj, grad i država institucije), e-mail adresa te ovisno o odabiru uloge autore rada (sudionici)
					\item prilikom prijave zatražiti novu lozinku
					
				\end{packed_enum}
			
				\item  \underbar{Sudionik (inicijator) može:}
				
				\begin{packed_enum}
					
					\item pregledavati i mijenjati osobne podatke (osim e-maila i lozinke)
					\item učitati rad u .pdf formatu (do kraja roka za predaju)
					\item po potrebi priložiti izmijenjenu verziju rada i pri tome obavijestiti recenzenta u učinjenome
					
					
				\end{packed_enum}
				
				\item  \underbar{Recenzent (inicijator) može:}
				
				\begin{packed_enum}
					
					\item pregledavati i mijenjati osobne podatke (osim e-maila i lozinke)
					\item dohvaćati pristigle radove
					\item preuzeti radove na lokalno računalo
					\item prihvatiti rad bez izmjena
					\item prihvatiti rad uz manje izmjene, bez naknadnih provjera
					\item odobriti rad uz veće izmjene, pri čemu mora obavijestiti autora te ponovno pregledati izmijenjeni rad
					\item u potpunosti odbiti rad, pri čemu je potrebno navesti razloge i objasniti ocjenu
					
					
				\end{packed_enum}
				
				\item  \underbar{Predsjedavajući konferencije (inicijator) može:}
				
				\begin{packed_enum}
				
					\item vidjeti sve podatke o sudionicima, mijenjati ih i dodavati sadržaj
					\item preuzeti sve pristigle radove na lokalno računalo
					\item dati odobrenje za obavljanje recenzije radova recenzentima
					\item svima ili samo odabranim korisnicima slati obavijesti na e-mail
				
				\end{packed_enum}
				
				\item  \underbar{Administrator (inicijator) može:}
				
				\begin{packed_enum}
				
					\item upisati podatke o konferenciji
					\item odrediti predsjedavajuče članove konferencije
				\end{packed_enum}

				\item \underbar{Baza podataka (sudionik):}

				\begin{packed_enum}

					\item pohranjuje podatke o korisnicima te njihovim ulogama i ovlastima u sustavu
					\item pohranjuje podatke o radovima, njihovim autorima i recenzijama tih radova

				\end{packed_enum}
					
			\end{packed_enum}
			
			\eject 
			
			
				
			\subsection{Obrasci uporabe}
				
				\textbf{\textit{dio 1. revizije}}
				
				\subsubsection{Opis obrazaca uporabe}
					
					\noindent \underbar{\textbf{UC1 - Pregled naslovnice}}
					\begin{packed_item}
	
						\item \textbf{Glavni sudionik: } Neprijavljeni korisnik, sudionik, recenzent
						\item  \textbf{Cilj:} Pregledavanje statičnog sadržaja naslovnice
						\item  \textbf{Sudionici:} Baza podataka
						\item  \textbf{Preduvjet:} -
						\item  \textbf{Opis osnovnog tijeka:}
						
						\item[] \begin{packed_enum}
	
							\item Na naslovnici se prikazuje statičan sadržaj, odnosno činjenice i informacije o konferenciji i sl.
							\item Neprijavljeni korisnici te prijavljeni sudionici i recenzenti mogu pregledavati spomenute sadržaje
						\end{packed_enum}
						
%ovaj dio je ostavljen iz razloga da ga se moze prekopirat za ostale obrasce uporabe
						\item  \textbf{Opis mogućih odstupanja:}
						
						\item[] \begin{packed_item}
	
							\item[2.a] $<$opis mogućeg scenarija odstupanja u koraku 2$>$
							\item[] \begin{packed_enum}
								
								\item $<$opis rješenja mogućeg scenarija korak 1$>$
								\item $<$opis rješenja mogućeg scenarija korak 2$>$
								
							\end{packed_enum}
							\item[2.b] $<$opis mogućeg scenarija odstupanja u koraku 2$>$
							\item[3.a] $<$opis mogućeg scenarija odstupanja  u koraku 3$>$
							
						\end{packed_item}
					\end{packed_item}

					\noindent \underbar{\textbf{UC2 - Registracija}}
					\begin{packed_item}
	
						\item \textbf{Glavni sudionik: } Neprijavljeni korisnik
						\item  \textbf{Cilj:} Stvaranje novog korisničkog računa za pristup aplikaciji
						\item  \textbf{Sudionici:} Baza podataka
						\item  \textbf{Preduvjet:} -
						\item  \textbf{Opis osnovnog tijeka:}
						
						\item[] \begin{packed_enum}
	
							\item Korisnik odabire opciju za registraciju
							\item Korisnik unosi osobne podatke potrebne za registraciju
							\item Nakon upisanih osobnih podataka, korisnik odabire kreira li račun kao sudionik ili recenzent, te ukoliko odabere opciju sudionik pojavljuje se još jedan dio obrasca gdje je potrebno unijeti naziv rada koji želi predati i autore istog
							\item Nakon registracije i nakon što administrator odobri prijavu, korisnik prima obavijest (putem e-maila) o uspješnoj registraciji u kojoj se nalazi lozinka s kojom se prijavljuje u sustav
						\end{packed_enum}

						\item  \textbf{Opis mogućih odstupanja:}
						
						\item[] \begin{packed_item}
	
							\item[2.a]  Odabir već zauzetog korisničkog imena i/ili e-maila, unos jednog ili više korisničkih podataka u nedozvoljenom formatu
							\item[] \begin{packed_enum}
								
								\item Sustav korisniku daje obavijest o tome gdje je nastala greška pri registraciji
								\item Korisnik sukladno mijenja unesene podatke sve dok nisu svi potrebni podaci ispravno upisani, odnosno u odgovarajućem formatu
								
							\end{packed_enum}
							
						\end{packed_item}
			
					\end{packed_item}
				
					
				\subsubsection{Dijagrami obrazaca uporabe}
					
					\textit{Prikazati odnos aktora i obrazaca uporabe odgovarajućim UML dijagramom. Nije nužno nacrtati sve na jednom dijagramu. Modelirati po razinama apstrakcije i skupovima srodnih funkcionalnosti.}
				\eject		
				
			\subsection{Sekvencijski dijagrami}
				
				\textbf{\textit{dio 1. revizije}}\\
				
				\textit{Nacrtati sekvencijske dijagrame koji modeliraju najvažnije dijelove sustava (max. 4 dijagrama). Ukoliko postoji nedoumica oko odabira, razjasniti s asistentom. Uz svaki dijagram napisati detaljni opis dijagrama.}
				\eject
	
		\section{Ostali zahtjevi}
		
			\textbf{\textit{dio 1. revizije}}\\
		 
			 \textit{Nefunkcionalni zahtjevi i zahtjevi domene primjene dopunjuju funkcionalne zahtjeve. Oni opisuju \textbf{kako se sustav treba ponašati} i koja \textbf{ograničenja} treba poštivati (performanse, korisničko iskustvo, pouzdanost, standardi kvalitete, sigurnost...). Primjeri takvih zahtjeva u Vašem projektu mogu biti: podržani jezici korisničkog sučelja, vrijeme odziva, najveći mogući podržani broj korisnika, podržane web/mobilne platforme, razina zaštite (protokoli komunikacije, kriptiranje...)... Svaki takav zahtjev potrebno je navesti u jednoj ili dvije rečenice.}
			 
			 
			 
	