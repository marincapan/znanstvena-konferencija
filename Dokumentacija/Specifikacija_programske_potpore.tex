\chapter{Specifikacija programske potpore}
		
	\section{Funkcionalni zahtjevi}
			
			\textbf{\textit{dio 1. revizije}}\\
			
			
			\noindent \textbf{Dionici:}
			
			\begin{packed_enum}
				
				\item Sudionici konferencije
				\item Recenzenti
				\item Predsjedavajući konferencije				
				\item Administrator sustava
				
			\end{packed_enum}
			
			\noindent \textbf{Aktori i njihovi funkcionalni zahtjevi:}
			
			
			\begin{packed_enum}
				\item  \underbar{Neregistriran/neprijavljen korisnik (inicijator) može:}
				
				\begin{packed_enum}
					
					\item pristupiti naslovnici konferencije i sadržaju na njoj, kao i stranici s informacijama o samoj konferenciji
					\item prijaviti se u sustav, a ako to još nije napravio, može se registrirati - stvoriti novi korisnički račun za koji su mu potrebni osobni podaci - ime i prezime, naziv matične ustanove (kao i ulica i kućni broj, grad i država institucije), e-mail adresa te ovisno o odabiru uloge autore rada (sudionici) te e-mail adresu autora koji je osoba za kontakt
					\item prilikom prijave zatražiti novu lozinku
					
				\end{packed_enum}
			
				\item  \underbar{Sudionik (inicijator) može:}
				
				\begin{packed_enum}
					
					\item pregledavati i mijenjati osobne podatke (osim e-maila i lozinke)
					\item učitati rad u \textit{.pdf} formatu (do kraja roka za predaju)
					\item po potrebi priložiti izmijenjenu verziju rada i pri tome obavijestiti recenzenta u učinjenome
					
					
				\end{packed_enum}
				
				\item  \underbar{Recenzent (inicijator) može:}
				
				\begin{packed_enum}
					
					\item pregledavati i mijenjati osobne podatke (osim e-mail adrese i lozinke)
					\item dohvaćati pristigle radove
					\item preuzeti radove na lokalno računalo
					\item prihvatiti rad bez izmjena
					\item prihvatiti rad uz manje izmjene, bez naknadnih provjera
					\item odobriti rad uz veće izmjene, pri čemu mora obavijestiti autora te ponovno pregledati izmijenjeni rad
					\item u potpunosti odbiti rad, pri čemu je potrebno navesti razloge i objasniti ocjenu
					
					
				\end{packed_enum}
				
				\item  \underbar{Predsjedavajući konferencije (inicijator) može:}
				
				\begin{packed_enum}
				
					\item vidjeti sve podatke o sudionicima, mijenjati ih i dodavati sadržaj
					\item preuzeti sve pristigle radove na lokalno računalo
					\item dati odobrenje za obavljanje recenzije radova recenzentima
					\item svima ili samo odabranim korisnicima slati obavijesti na e-mail
				
				\end{packed_enum}
				
				\item  \underbar{Administrator (inicijator) može:}
				
				\begin{packed_enum}
				
					\item upisati podatke o konferenciji
					\item odrediti predsjedavajućeg konferencije
				\end{packed_enum}

				\item \underbar{Baza podataka (sudionik):}

				\begin{packed_enum}

					\item pohranjuje podatke o korisnicima te njihovim ulogama i ovlastima u sustavu
					\item pohranjuje podatke o radovima, njihovim autorima i recenzijama tih radova

				\end{packed_enum}
					
			\end{packed_enum}
			
			\eject 
			
			
				
			\subsection{Obrasci uporabe}
				
				\textbf{\textit{dio 1. revizije}}
				
				\subsubsection{Opis obrazaca uporabe}
					
					\noindent \underbar{\textbf{UC1 - Pregled naslovnice}}
					\begin{packed_item}
	
						\item \textbf{Glavni sudionik: } Neprijavljeni korisnik, sudionik, recenzent
						\item  \textbf{Cilj:} Pregledavanje statičnog sadržaja naslovnice
						\item  \textbf{Sudionici:} Baza podataka
						\item  \textbf{Preduvjet:} -
						\item  \textbf{Opis osnovnog tijeka:}
						
						\item[] \begin{packed_enum}
	
							\item Na naslovnici se prikazuje statičan sadržaj, odnosno činjenice i informacije o konferenciji i sl.
							\item Neprijavljeni korisnici te prijavljeni sudionici i recenzenti mogu pregledavati spomenute sadržaje
						\end{packed_enum}
						

					\end{packed_item}

					\noindent \underbar{\textbf{UC2 - Registracija}}
					\begin{packed_item}
	
						\item \textbf{Glavni sudionik: } Neprijavljeni korisnik
						\item  \textbf{Cilj:} Stvaranje novog korisničkog računa za pristup aplikaciji
						\item  \textbf{Sudionici:} Baza podataka
						\item  \textbf{Preduvjet:} -
						\item  \textbf{Opis osnovnog tijeka:}
						
						\item[] \begin{packed_enum}
	
							\item Korisnik odabire opciju za registraciju
							\item Korisnik unosi osobne podatke potrebne za registraciju
							\item Nakon upisanih osobnih podataka, korisnik odabire kreira li račun kao sudionik ili recenzent, te ukoliko odabere opciju sudionik pojavljuje se još jedan dio obrasca gdje je potrebno unijeti naziv rada koji želi predati i autore istog
							\item Ažurira se baza podataka
							\item Nakon registracije i nakon što administrator odobri prijavu, korisnik prima obavijest (putem e-maila) o uspješnoj registraciji u kojoj se nalazi lozinka s kojom se prijavljuje u sustav
						\end{packed_enum}

						\item  \textbf{Opis mogućih odstupanja:}
						
						\item[] \begin{packed_item}
	
							\item[2.a]  Odabir već zauzetog korisničkog imena i/ili e-maila, unos jednog ili više korisničkih podataka u nedozvoljenom formatu
							\item[] \begin{packed_enum}
								
								\item Sustav korisniku daje obavijest o tome gdje je nastala greška pri registraciji
								\item Korisnik sukladno mijenja unesene podatke sve dok nisu svi potrebni podaci ispravno upisani, odnosno u odgovarajućem formatu
								
							\end{packed_enum}
							
						\end{packed_item}
			
					\end{packed_item}

					\noindent \underbar{\textbf{UC3 - Prijava u sustav}}
					\begin{packed_item}
	
						\item \textbf{Glavni sudionik: } Sudionik konferencije, recenzent, administrator, predsjedavajući
						\item  \textbf{Cilj:} Dobiti pristup korisničkom sučelju
						\item  \textbf{Sudionici:} Baza podataka
						\item  \textbf{Preduvjet:} Registracija
						\item  \textbf{Opis osnovnog tijeka:}
						
						\item[] \begin{packed_enum}
	
							\item Unos korisničkog imena i lozinke
							\item Potvrda o ispravnosti unesenih podataka
							\item Pristup korisničkom sučelju
					
						\end{packed_enum}

						\item  \textbf{Opis mogućih odstupanja:}
						
						\item[] \begin{packed_item}
	
							\item[2.a]  Neispravno korisničko ime ili lozinka
							\item[] \begin{packed_enum}
								
								\item Sustav korisniku daje obavijest o neispravno unesenim podacima, ponovno učitava stranicu za prijavu te upućuje korisnika da ukoliko nije izvršena registracija prvo napravi taj korak
								
							\end{packed_enum}
							
						\end{packed_item}
			
					\end{packed_item}

					\noindent \underbar{\textbf{UC4 - Pregled osobnih podataka}}
					\begin{packed_item}
	
						\item \textbf{Glavni sudionik: } Sudionik konferencije, recenzent
						\item  \textbf{Cilj:} Pregledati osobne podatke
						\item  \textbf{Sudionici:} Baza podataka
						\item  \textbf{Preduvjet:} Prijava u sustav
						\item  \textbf{Opis osnovnog tijeka:}
						
						\item[] \begin{packed_enum}
	
							\item Korisnik odabere opciju "Osobni podaci"
							\item Aplikacija prikazuje osobne podatke o korisniku povučene iz baze podataka
					
						\end{packed_enum}
			
					\end{packed_item}

					\noindent \underbar{\textbf{UC5 - Promjena osobnih podataka}}
					\begin{packed_item}
	
						\item \textbf{Glavni sudionik: } Sudionik konferencije, recenzent
						\item  \textbf{Cilj:} Izmijeniti osobne podatke
						\item  \textbf{Sudionici:} Baza podataka
						\item  \textbf{Preduvjet:} Prijava u sustav
						\item  \textbf{Opis osnovnog tijeka:}
						
						\item[] \begin{packed_enum}
	
							\item Korisnik odabere opciju "Promijeniti" na stranici za pregled osobnih podataka
							\item Korisnik radi željene izmjene
							\item Korisnik odabire opciju "Spremi promjene"
							\item Baza podataka se ažurira
					
						\end{packed_enum}

						\item  \textbf{Opis mogućih odstupanja:}
						
						\item[] \begin{packed_item}
	
							\item[2.a]  Korisnik izvrši promjenu podataka, ali uneseni podaci nisu u ispravnom formatu
							\item[] \begin{packed_enum}
								
								\item Sustav ne dozvoljava spremanje izmjena dok podaci nisu u ispravnom formatu
								\item Sustav kraj podatka koji je neispravno zapisan prikaže obavijest kako podatak nije u ispravnom formatu te upućuje da korisnik ispravi isto
								
							\end{packed_enum}

							\item[2.b]  Korisnik izvrši promjenu podataka, ali ne odabere opciju "Spremi promjene"
							\item[] \begin{packed_enum}
								
								\item Sustav obavještava korisnika da nije spremio podatke prije izlaska iz prozora
								
							\end{packed_enum}
							
						\end{packed_item}
			
					\end{packed_item}

					\noindent \underbar{\textbf{UC6 - Predaja rada}}
					\begin{packed_item}
	
						\item \textbf{Glavni sudionik: } Sudionik konferencije
						\item  \textbf{Cilj:} Predati rad
						\item  \textbf{Sudionici:} Baza podataka
						\item  \textbf{Preduvjet:} Prijava u sustav
						\item  \textbf{Opis osnovnog tijeka:}
						
						\item[] \begin{packed_enum}
	
							\item Na stranici "Moji radovi" korisnik odabire opciju "Predaj rad"
							\item Pojavljuju se već ispunjena polja s nazivom rada i skupom autora koje je korisnik pri registraciji naveo, korisnik odabire opciju „Učitaj rad“.
							\item Korisnik učitava rad u .pdf formatu
							\item Korisnik odabire opciju "Zaključi predaju"
							\item Rad se pohranjuje i baza podataka se ažurira

					
						\end{packed_enum}

						\item  \textbf{Opis mogućih odstupanja:}
						
						\item[] \begin{packed_item}
	
							\item[3.a]  Korisnik je učitao datoteku koja nije u *.pdf obliku
							\item[] \begin{packed_enum}
								
								\item Sustav javlja da je dozvoljeno učitavati isključivo .pdf dokumente i ne pohranjuje ono što je korisnik pokušao prenijeti
								
							\end{packed_enum}

							\item[3.b]  Korisnik prenese datoteku, ali ne odabere opciju „Zaključi predaju“
							\item[] \begin{packed_enum}
								
								\item Sustav javlja korisniku kako nije pohranio rad prije izlaska iz prozora
								
							\end{packed_enum}
							
						\end{packed_item}
			
					\end{packed_item}

					\noindent \underbar{\textbf{UC7 - Zahtjev za predaju dodatnih radova}}
					\begin{packed_item}
	
						\item \textbf{Glavni sudionik: } Sudionik konferencije
						\item  \textbf{Cilj:} Predati zahtjev kako bi korisnik mogao predati još jedan rad
						\item  \textbf{Sudionici:} Baza podataka
						\item  \textbf{Preduvjet:} Prijava u sustav, prethodno predan minimalno jedan rad
						\item  \textbf{Opis osnovnog tijeka:}
						
						\item[] \begin{packed_enum}
	
							\item Na stranici "Moji radovi" korisnik odabire opciju "Želim predati još jedan rad"
							\item Korisnik ispunjava formu za predaju zahtjeva
							\item Korisnik odabire opciju "Predaj zahtjev"
							\item Korisnik dobiva odgovor na zahtjev putem e-maila

					
						\end{packed_enum}
			
					\end{packed_item}
					\noindent \underbar{\textbf{UC8 - Predavanje dodatnih radova nakon inicijalnog}}
					\begin{packed_item}
						
						\item \textbf{Glavni sudionik:} Sudionik konferencije
						\item \textbf{Cilj:} Učitati još jedan rad u sustav
						\item \textbf{Sudionici:} Baza podataka
						\item \textbf{Preduvjet:} Dobiven potvrdan odgovor na zahtjev za predaju dodatnih radova
						\item \textbf{Opis osnovnog tijeka:}
						
						\item[] \begin{packed_enum}
							
							\item Na stranici "Moji radovi" korisnik odabire opciju "Želim predati još jedan rad"
							\item Pojavljuju se već ispunjena polja s nazivom rada i skupom autora koje je korisnik pri predavanju zahtjeva za dodatan rad naveo, korisnik odabire opciju "Učitaj rad"
							\item Korisnik učitava rad u .pdf formatu
							\item Korisnik odabire opciju "Zaključi predaju"
							\item Rad se pohranjuje i baza podataka se ažurira
							
						\end{packed_enum} 
					
					\item \textbf{Opis mogućih odstupanja:}
					
					\item[] \begin{packed_item}
						\item[3.a] Korisnik je učitao datoteku koja nije u .pdf formatu
						
						\item [] \begin{packed_enum}
							\item Sustav javlja da je dozvoljeno učitavati isključivo .pdf dokumente i ne pohranjuje ono što je korisnik pokušao prenijeti
							
						\end{packed_enum}
					
						\item[3.b] Korisnik prenese datoteku, ali ne odabere opciju "Zaključi predaju"
						
						\item[] \begin{packed_enum}
							\item Sustav javlja korisniku kako nije pohranio rad prije izlaska iz prozora
						\end{packed_enum}
					
					\end{packed_item}
						
					\end{packed_item}

					\noindent \underbar{\textbf{UC9 - Odobravanje registracije sudionika}}
					\begin{packed_item}
	
						\item \textbf{Glavni sudionik: } Administrator
						\item  \textbf{Cilj:} Odobriti registraciju sudionika i poslati mu podatke za prijavu
						\item  \textbf{Sudionici:} Baza podataka
						\item  \textbf{Preduvjet:} Prijava u sustav
						\item  \textbf{Opis osnovnog tijeka:}
						
						\item[] \begin{packed_enum}
	
							\item Korisnik odabire popis registracija sudionika koje čekaju odobrenje
							\item Korisnik odabire registraciju s popisa
							\item Korisniku se prikazuju podaci koje je sudionik unio pri registraciji
							\item Korisnik odabire opciju "Potvrdi prijavu i pošalji potvrdu"
							\item Šalje se automatska poruka putem mail-a odabranom sudioniku
							\item Ažurira se baza podataka

					
						\end{packed_enum}
			
					\end{packed_item}

					\noindent \underbar{\textbf{UC10 - Odobravanje registracije recenzenta}}
					\begin{packed_item}
	
						\item \textbf{Glavni sudionik: } Predsjedavajući konferencije
						\item  \textbf{Cilj:} Odobriti registraciju recenzenta i poslati mu podatke za prijavu
						\item  \textbf{Sudionici:} Baza podataka
						\item  \textbf{Preduvjet:} Prijava u sustav
						\item  \textbf{Opis osnovnog tijeka:}
						
						\item[] \begin{packed_enum}
	
							\item Korisnik odabire popis registracija recenzenta koje čekaju odobrenje
							\item Korisnik odabire registraciju s popisa
							\item Korisniku se prikazuju podaci koje je recenzent unio pri registraciji
							\item Korisnik odabire opciju "Potvrdi prijavu i pošalji potvrdu"
							\item Šalje se automatska poruka putem mail-a odabranom recenzentu
							\item Ažurira se baza podataka

					
						\end{packed_enum}
			
					\end{packed_item}

					\noindent \underbar{\textbf{UC11 - Odbijanje registracije sudionika}}
					\begin{packed_item}
	
						\item \textbf{Glavni sudionik: } Administrator
						\item  \textbf{Cilj:} Odbiti registraciju sudionika i izbrisati ga iz baze podataka
						\item  \textbf{Sudionici:} Baza podataka
						\item  \textbf{Preduvjet:} Prijava u sustav
						\item  \textbf{Opis osnovnog tijeka:}
						
						\item[] \begin{packed_enum}
	
							\item Korisnik odabire popis registracija sudionika koje čekaju odobrenje
							\item Korisnik odabire registraciju s popisa
							\item Korisniku se prikazuju podaci koje je sudionik unio pri registraciji
							\item Korisnik odabire opciju "Odbij prijavu"
							\item Otvara se polje za unos obrazloženja odbijanja prijave
							\item Korisnik navodi razlog/e i odabire opciju "Pošalji odbijenicu"
							\item Šalje se personalizirana poruka putem mail-a odabranom sudioniku
							\item Ažurira se baza podataka

					
						\end{packed_enum}
			
					\end{packed_item}

					\noindent \underbar{\textbf{UC12 - Odbijanje registracije recenzenta}}
					\begin{packed_item}
	
						\item \textbf{Glavni sudionik: } Predsjedavajući konferencije
						\item  \textbf{Cilj:} Odbiti registraciju recenzenta i izbrisati ga iz baze podataka
						\item  \textbf{Sudionici:} Baza podataka
						\item  \textbf{Preduvjet:} Prijava u sustav
						\item  \textbf{Opis osnovnog tijeka:}
						
						\item[] \begin{packed_enum}
	
							\item Korisnik odabire popis registracija recenzenta koje čekaju odobrenje
							\item Korisnik odabire registraciju s popisa
							\item Korisniku se prikazuju podaci koje je recenzent unio pri registraciji
							\item Korisnik odabire opciju "Odbij prijavu"
							\item Otvara se polje za unos obrazloženja odbijanja prijave
							\item Korisnik navodi razlog/e i odabire opciju "Pošalji odbijenicu"
							\item Šalje se personalizirana poruka putem mail-a odabranom recenzentu
							\item Ažurira se baza podataka

					
						\end{packed_enum}
			
					\end{packed_item}

					\noindent \underbar{\textbf{UC13 - Pregled sudionika}}
					\begin{packed_item}
	
						\item \textbf{Glavni sudionik: } Administrator, predsjedavajući konferencije
						\item  \textbf{Cilj:} Pregledati popis registriranih i potvrđenih sudionika
						\item  \textbf{Sudionici:} Baza podataka
						\item  \textbf{Preduvjet:} Prijava u sustav
						\item  \textbf{Opis osnovnog tijeka:}
						
						\item[] \begin{packed_enum}
	
							\item Korisnik odabire opciju "Popis registriranih sudionika"
							\item Korisniku se prikazuje popis registriranih sudionika iz baze podataka
							\item Korisnik odabire sudionika s popisa
							\item Korisniku se prikazuju informacije o odabranom sudioniku

					
						\end{packed_enum}
			
					\end{packed_item}

					\noindent \underbar{\textbf{UC14 - Pregled recenzenata}}
					\begin{packed_item}
	
						\item \textbf{Glavni sudionik: } Administrator, predsjedavajući konferencije
						\item  \textbf{Cilj:} Pregledati popis registriranih i potvrđenih recezenata
						\item  \textbf{Sudionici:} Baza podataka
						\item  \textbf{Preduvjet:} Prijava u sustav
						\item  \textbf{Opis osnovnog tijeka:}
						
						\item[] \begin{packed_enum}
	
							\item Korisnik odabire opciju "Popis registriranih recenzenata"
							\item Korisniku se prikazuje popis registriranih recenzenata iz baze podataka
							\item Korisnik odabire recenzenta s popisa
							\item Korisniku se prikazuju informacije o odabranom recenzentu

					
						\end{packed_enum}
			
					\end{packed_item}

					\noindent \underbar{\textbf{UC15 - Dodavanje novog administratora}}
					\begin{packed_item}
	
						\item \textbf{Glavni sudionik: } Administrator
						\item  \textbf{Cilj:} Dodati novog administratora u bazu podataka
						\item  \textbf{Sudionici:} Baza podataka
						\item  \textbf{Preduvjet:} Prijava u sustav
						\item  \textbf{Opis osnovnog tijeka:}
						
						\item[] \begin{packed_enum}
	
							\item Korisnik pristupa administracijskom sučelju
							\item Korisnik bira opciju "Dodaj novog administratora"
							\item Korisnik unosi osobne podatke administratora kojeg želi dodati
							\item Korisnik odabire opciju "Dodaj administratora u sustav"
							\item Ažurira se baza podataka

					
						\end{packed_enum}

						\item  \textbf{Opis mogućih odstupanja:}
						
						\item[] \begin{packed_item}
	
							\item[3.a]  Korisnik unese podatke o administratoru, ali ne odabere opciju "Dodaj administratora u sustav"
							\item[] \begin{packed_enum}
								
								\item Sustav javlja korisniku kako novi administrator nije dodan u sustav prije izlaska iz prozora
								
							\end{packed_enum}
							
						\end{packed_item}
			
					\end{packed_item}
				
				
					\noindent \underbar{\textbf{UC16 - Recenziranje rada}}
					\begin{packed_item}
						
						\item \textbf{Glavni sudionik:} Recenzent
						\item \textbf{Cilj:} Pregledati i ocijeniti rad
						\item \textbf{Sudionici:} Baza podataka
						\item \textbf{Preduvjet:} Prijava u sustav
						\item \textbf{Opis osnovnog tijeka:} 
						
						\item[] \begin{packed_enum}
							\item Na stranici "Moji radovi" korisnik odabire jedan od ponuđenih radova za recenziju
							\item  Korisnik na stranici vidi prikaz dokumenta kojeg recenzira (uz opciju "Preuzmi rad" koja pohranjuje .pdf lokalno)
							\item Korisnik bira jednu od 4 moguće ocjene te ovisno o odabranoj ocjeni u polje za obrazloženje navodi razlog svog odabira
							\item Korisnik odabire opciju "Zaključaj recenziju i obavijesti autora/e rada"
							\item Recenzija se pohranjuje i šalje autoru/ima rada na e-mail i baza podataka se ažurira
						\end{packed_enum}
					
						\item \textbf{Opis mogućih odstupanja:}
						
						\item[] \begin{packed_enum}
							\item[3.a] Korisnik nije odabrao ocjenu ili upisao obrazloženje
							\item[] \begin{packed_enum}
								\item[1.] Sustav javlja da se ocjena mora obrazloženje i da komentar mora biti upisano
							\end{packed_enum}
							\item[3.b] Korisnik je odabrao ocjenu i upisao obrazloženje, ali nije odabrao opciju "Pošalji recenziju"
							\item[] \begin{packed_enum}
								\item[1.] Sustav javlja korisniku kako nije pohranio recenziju prije izlaska iz prozora
							\end{packed_enum}
						\end{packed_enum}
					
					\end{packed_item}
				
					\noindent \underbar{\textbf{UC17 - Izmjena rada po zahtjevu recenzenta}}
					\begin{packed_item}
						\item \textbf{Glavni sudionik:} Sudionik
						\item \textbf{Cilj:} Izmijeniti i učitati ranije učitan rad
						\item \textbf{Sudionici:} Baza podataka
						\item \textbf{Preduvjet:} Prijava u sustav, recenzent dao ocjenu za koju je rad potrebno izmijeniti
						
						\item \textbf{Opis osnovnog tijeka:} 
						\item[] \begin{packed_enum}
							\item Na stranici "Moji radovi" korisnik odabire rad koji želi izmijeniti
							\item Korisnik učitava izmijenjeni rad u .pdf formatu
							\item Korisnik odabire opciju "Zaključi predaju"
							\item Ako je rad ocijenjen da su potrebne veće izmjene, automatski se šalje e-mail recenzentu rada o obavljenoj izmjeni
							\item Rad se pohranjuje i baza podataka se ažurira
						\end{packed_enum}
					
						\item \textbf{Opis mogućih odstupanja:}
						\item[] \begin{packed_enum}
							\item[2.a] Korisnik je učitao datoteku koja nije u .pdf formatu
							\item[] \begin{packed_enum}
								\item[1.] Sustav javlja da je dozvoljeno učitati isključivo .pdf dokumente i ne pohranjuje ono što je korisnik pokušao prenijeti
							\end{packed_enum}
						
							\item[2.b] Korisnik prenese datoteku, ali ne odabere opciju "Zaključi predaju"
							\item[] \begin{packed_enum}
								\item[1.] Sustav javlja korisniku kako nije pohranio rad prije izlaska iz prozora
							\end{packed_enum}
							
						\end{packed_enum}
					\end{packed_item}
					
					\noindent \underbar{\textbf{UC18 - Pregled radova}}
					\begin{packed_item}
	
						\item \textbf{Glavni sudionik: } Administrator
						\item  \textbf{Cilj:} Pregledati popis predanih radova
						\item  \textbf{Sudionici:} Baza podataka
						\item  \textbf{Preduvjet:} Prijava u sustav
						\item  \textbf{Opis osnovnog tijeka:}
						
						\item[] \begin{packed_enum}
	
							\item Korisnik odabire opciju "Popis predanih radova"
							\item Korisniku se prikazuje popis predanih radova
							\item Korisnik odabire rad s popisa
							\item Korisniku se prikazuju informacije o odabranom radu
							\item Korisnik, ukoliko on to želi, odabire opciju "Preuzmi rad" te preuzima rad lokalno sebi na računalo

					
						\end{packed_enum}
			
					\end{packed_item}
				
					\noindent \underbar{\textbf{UC19 - Učini radove javno dostupnima}}
					
					\begin{packed_item}
						
						\item \textbf{Glavni sudionik: } Administrator
						\item  \textbf{Cilj:} Učiniti radove javno dostupnima
						\item  \textbf{Sudionici:} -
						\item  \textbf{Preduvjet:} Prijava u sustav, nedozvoljeno visoka vrijednost indeksa Covid-19 zaraze u državi konferencije
						\item  \textbf{Opis osnovnog tijeka:}
						
						\item[] \begin{packed_enum}
							
							\item Korisnik odabire opciju "Popis predanih radova"
							\item Korisnik odabire gumb "Učini radove javno dostupnima"
							\item Korisniku se pojavljuje pitanje "Jeste li sigurni da želite javno objaviti radove?", korisnik odabire "Da"
							\item Korisniku se prikazuje poruka o uspješnoj objavi radova
							
						\end{packed_enum}
					
						\item \textbf{Opis mogućih odstupanja:}
						\item[] \begin{packed_enum}
							\item[3.a] Korisnik odabire gumb "Ne" umjesto "Da"
							\item[] \begin{packed_enum}
								\item[1.] Korisnik je vraćen na stranicu popisa predanih radova
							\end{packed_enum} 
						\end{packed_enum}
						
					\end{packed_item}

					
					
				
					
				\subsubsection{Dijagrami obrazaca uporabe}
					
					\textit{Prikazati odnos aktora i obrazaca uporabe odgovarajućim UML dijagramom. Nije nužno nacrtati sve na jednom dijagramu. Modelirati po razinama apstrakcije i skupovima srodnih funkcionalnosti.}
				\eject		
				
			\subsection{Sekvencijski dijagrami}
				
				\textbf{\textit{dio 1. revizije}}\\
				
				\textit{Nacrtati sekvencijske dijagrame koji modeliraju najvažnije dijelove sustava (max. 4 dijagrama). Ukoliko postoji nedoumica oko odabira, razjasniti s asistentom. Uz svaki dijagram napisati detaljni opis dijagrama.}
				\eject
	
		\section{Ostali zahtjevi}
		
			\textbf{\textit{dio 1. revizije}}\\
		 
			 \textit{Nefunkcionalni zahtjevi i zahtjevi domene primjene dopunjuju funkcionalne zahtjeve. Oni opisuju \textbf{kako se sustav treba ponašati} i koja \textbf{ograničenja} treba poštivati (performanse, korisničko iskustvo, pouzdanost, standardi kvalitete, sigurnost...). Primjeri takvih zahtjeva u Vašem projektu mogu biti: podržani jezici korisničkog sučelja, vrijeme odziva, najveći mogući podržani broj korisnika, podržane web/mobilne platforme, razina zaštite (protokoli komunikacije, kriptiranje...)... Svaki takav zahtjev potrebno je navesti u jednoj ili dvije rečenice.}
			 
			 
			 
	