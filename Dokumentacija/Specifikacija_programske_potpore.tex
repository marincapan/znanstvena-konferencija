\chapter{Specifikacija programske potpore}
		
	\section{Funkcionalni zahtjevi}
			
			\textbf{\textit{dio 1. revizije}}\\
			
			
			\noindent \textbf{Dionici:}
			
			\begin{packed_enum}
				
				\item Znanstvena udruga "Pametna ekipa" (naručitelj)
				\item Sudionici konferencije
				\item Recenzenti
				\item Predsjedavajući konferencije				
				\item Administrator(i) sustava
				\item Razvojni tim
				
			\end{packed_enum}
			
			\noindent \textbf{Aktori i njihovi funkcionalni zahtjevi:}
			
			
			\begin{packed_enum}
					\item  \underbar{ Neregistrirani/neprijavljeni korisnik (inicijator) može:}
				
				\begin{packed_enum}
					
					\item pristupiti naslovnici konferencije i sadržaju na njoj, kao i stranici s informacijama o samoj konferenciji
					\item registrirati se - stvoriti novi korisnički račun za koji su mu potrebni osobni podaci - ime i prezime, naziv matične ustanove (kao i ulica i kućni broj, grad i država institucije), e-mail adresa; korisnik koji se registrira kao sudionik konferencije unosi i naslov rada, autore te e-mail adresu autora koji je osoba za kontakt 
					\item prijaviti se u sustav (ukoliko se prethodno registrirao)
					\item na stranici prijave zatražiti novu lozinku (ukoliko se prethodno registrirao)
					
				\end{packed_enum}
			 	\item  \underbar{Registrirani korisnik (inicijator):}
			 
			 	\begin{packed_enum}
					\item  \underbar{Sudionik (inicijator) može:}
				
					\begin{packed_enum}
					
						\item pregledavati i mijenjati osobne podatke (osim e-maila i lozinke)
						\item učitati rad u \textit{.pdf} formatu (do kraja roka za predaju)
						\item po potrebi priložiti izmijenjenu verziju rada i pri tome obavijestiti recenzenta u učinjenome
						\item mijenjati podatke prijavnice
					
					
					\end{packed_enum}
				
					\item  \underbar{Recenzent (inicijator) može:}
				
					\begin{packed_enum}
					
						\item pregledavati i mijenjati osobne podatke (osim e-maila i lozinke)
						\item pregledavati pristigle radove
						\item preuzeti radove na lokalno računalo
						\item prihvatiti rad bez izmjena
						\item prihvatiti rad uz manje izmjene, bez naknadnih provjera
						\item odobriti rad uz veće izmjene, pri čemu mora obavijestiti autora te ponovno pregledati izmijenjeni rad
						\item u potpunosti odbiti rad, pri čemu je potrebno navesti razloge i objasniti ocjenu
					
					
					\end{packed_enum}
				
					\item  \underbar{Predsjedavajući konferencije (inicijator) može:}
				
					\begin{packed_enum}
				
						\item vidjeti sve podatke o sudionicima, mijenjati ih i dodavati sadržaj
						\item preuzeti sve pristigle radove na lokalno računalo
						\item dati odobrenje za obavljanje recenzije radova recenzentima
						\item svima ili samo odabranim korisnicima slati obavijesti na e-mail
					
				
					\end{packed_enum}
				
					\item  \underbar{Administrator (inicijator) može:}
				
					\begin{packed_enum}
				
						\item upisati podatke o konferenciji
						\item odrediti predsjedavajućeg konferencije
						\item pratiti stanje COVID-19 pandemije putem javno dostupnih podataka o incidenciji u pojedinim državama
						\item objaviti radove online u slučaju pogoršanja epidemiološke situacije
						\item definirati upitnik za prijavu na konferenciju
						\item vidjeti broj i imena aktivno registriranih korisnika
						\item pregledavati statistiku (broj radova) prijava po državama, sekcijama
						\item pregledavati i mijenjati podatke sudionika
						\item poslati obavijest pojedinim ili svim sudionicima 

					\end{packed_enum}
				\end{packed_enum}


				\item \underbar{Baza podataka (sudionik):}

				\begin{packed_enum}

					\item pohranjuje podatke o korisnicima i njihovim ovlastima u sustavu
					\item pohranjuje podatke o radovima, njihovim autorima i recenzijama tih radova
					\item pohranjuje podatke o konferenciji i upitniku za prijavu na konferenciju
					\item pohranjuje podatke o ustanovama i državama iz kojih dolaze registrirani sudionici

				\end{packed_enum}
					
			\end{packed_enum}
			
			\eject 
			
			
				
			\subsection{Obrasci uporabe}
				
				\textbf{\textit{dio 1. revizije}}
				
				\subsubsection{Opis obrazaca uporabe}
					
					\noindent \underbar{\textbf{UC1 - Pregled naslovnice}}
					\begin{packed_item}
	
						\item \textbf{Glavni sudionik: } Neprijavljeni korisnik, registrirani korisnik
						\item  \textbf{Cilj:} Pregledavanje statičnog sadržaja stranice
						\item  \textbf{Sudionici:} Baza podataka
						\item  \textbf{Preduvjet:} -
						\item  \textbf{Opis osnovnog tijeka:}
						
						\item[] \begin{packed_enum}
	
							\item Otvaranjem linka na stranici se prikazuje statičan sadržaj
							\item Pregledavanje sadržaja
						\end{packed_enum}
						

					\end{packed_item}

					\noindent \underbar{\textbf{UC2 - Registracija}}
					\begin{packed_item}
	
						\item \textbf{Glavni sudionik: } Neprijavljeni korisnik
						\item  \textbf{Cilj:} Stvaranje novog korisničkog računa za pristup aplikaciji
						\item  \textbf{Sudionici:} Baza podataka
						\item  \textbf{Preduvjet:} Korisnik nema račun
						\item  \textbf{Opis osnovnog tijeka:}
						
						\item[] \begin{packed_enum}
	
							\item Korisnik odabire opciju "Registracija" nakon čega se otvara forma za registraciju
							\item Korisnik ispunjava formu
							\item Korisnik odabire kreira li račun kao sudionik ili recenzent, te ukoliko odabere opciju sudionik pojavljuje se još jedan dio obrasca gdje je potrebno unijeti naziv rada koji želi predati i autore istog te označiti osobu za kontakt
							\item Sustav svakom sudioniku pridjeljuje jedinstveni identifikator ID\_xxxx, pri čemu je xxxx redni broj prijave u sustav
							\item Ažurira se baza podataka
							\item Korisnik koji se registrirao kao sudionik konferencije prima obavijest (putem e-maila) o uspješnoj registraciji (koju je potrebno potvrditi "klikom" na dobiveni link) u kojoj se nalazi lozinka i korisničko ime, a recenzent tu obavijest prima po odobrenju predsjedavajućeg
							
						\end{packed_enum}

						\item  \textbf{Opis mogućih odstupanja:}
						
						\item[] \begin{packed_item}
	
							\item[2.a]  Odabir već zauzetog e-maila, unos jednog ili više korisničkih podataka u nedozvoljenom formatu, postoji već rad s tim naslovom i istim autorima
							\item[] \begin{packed_enum}
								
								\item Sustav korisniku daje obavijest o tome gdje je nastala greška pri registraciji
								\item Korisnik sukladno mijenja unesene podatke sve dok nisu svi potrebni podaci ispravno upisani, odnosno u odgovarajućem formatu
								
							\end{packed_enum}
							
						\end{packed_item}
			
					\end{packed_item}

					\noindent \underbar{\textbf{UC3 - Prijava u sustav}}
					\begin{packed_item}
	
						\item \textbf{Glavni sudionik: } Registrirani korisnik
						\item  \textbf{Cilj:} Dobiti pristup korisničkom sučelju
						\item  \textbf{Sudionici:} Baza podataka
						\item  \textbf{Preduvjet:} Odrađena uspješna registracija
						\item  \textbf{Opis osnovnog tijeka:}
						
						\item[] \begin{packed_enum}

							\item Korisnik na naslovnici ili u glavnog izborniku bira opciju "Prijava", nakon čega se učitava stranica za prijavu
							\item Unos korisničkog imena i lozinke
							\item Korisnik može kliknuti gumb "Zaboravljena lozinka" te mu se generira nova (UC 3.1)
							\item Potvrda o ispravnosti unesenih podataka
							\item Pristup korisničkom sučelju
					
						\end{packed_enum}

						\item  \textbf{Opis mogućih odstupanja:}
						
						\item[] \begin{packed_item}
	
							\item[2.a]  Neispravno korisničko ime ili lozinka
							\item[] \begin{packed_enum}
								
								\item Sustav korisniku daje obavijest o neispravno unesenim podacima, ponovno učitava stranicu za prijavu te upućuje korisnika da ukoliko nije izvršena registracija prvo napravi taj korak
								
							\end{packed_enum}
							
						\end{packed_item}
			
					\end{packed_item}
				\noindent \underbar{\textbf{UC3.1 - Zahtjev za generiranjem nove lozinke}}
				\begin{packed_item}
					
					\item \textbf{Glavni sudionik: } Registrirani korisnik
					\item  \textbf{Cilj:} Dobiti novu lozinku
					\item  \textbf{Sudionici:} Baza podataka
					\item  \textbf{Preduvjet:} Korisnik je registriran
					\item  \textbf{Opis osnovnog tijeka:}
					
					\item[] \begin{packed_enum}
				
						\item Korisniku je učitana stranica za prijavu te je on kliknuo na gumb "Zaboravljena lozinka"
						\item Od korisnika se traži da unese e-mail na koju će mu biti poslana nova lozinka
						\item Generira se lozinka i šalje korisniku
						\item Aplikacija obavještava korisnika da mu je lozinka poslana na e-mail
						\item Ažurira se baza podataka
						
					\end{packed_enum}
					
					\item  \textbf{Opis mogućih odstupanja:}
					
					\item[] \begin{packed_item}
						
						\item[2.a]  Neispravan format e-maila i/ili e-mail ne postoji u bazi podataka
						\item[] \begin{packed_enum}
							
							\item Sustav korisniku daje obavijest o neispravno unesenim podacima, ponovno učitava stranicu za prijavu te upućuje korisnika da ukoliko nije izvršena registracija prvo napravi taj korak
							
						\end{packed_enum}
						
					\end{packed_item}
					
				\end{packed_item}

					\noindent \underbar{\textbf{UC4 - Pregled osobnih podataka}}
					\begin{packed_item}
	
						\item \textbf{Glavni sudionik: } Registrirani korisnik
						\item  \textbf{Cilj:} Pregledati osobne podatke
						\item  \textbf{Sudionici:} Baza podataka
						\item  \textbf{Preduvjet:} Korisnik je prijavljen u sustav
						\item  \textbf{Opis osnovnog tijeka:}
						
						\item[] \begin{packed_enum}
	
							\item Korisnik klikom na gumb "Moj profil" otvara stranicu svog profila
							\item Aplikacija prikazuje osobne podatke o korisniku povučene iz baze podataka
					
						\end{packed_enum}
			
					\end{packed_item}

					\noindent \underbar{\textbf{UC5 - Promjena osobnih podataka}}
					\begin{packed_item}
	
						\item \textbf{Glavni sudionik: } Registrirani korisnik
						\item  \textbf{Cilj:} Izmijeniti osobne podatke
						\item  \textbf{Sudionici:} Baza podataka
						\item  \textbf{Preduvjet:} Korisnik je prijavljen u sustav
						\item  \textbf{Opis osnovnog tijeka:}
						
						\item[] \begin{packed_enum}
							\item Korisnik odabire gumb "Moj profil" iz glavnog izbornika
							\item Korisnik odabere opciju "Promijeniti" kraj osobnih podataka
							\item Korisnik radi željene izmjene
							\item Korisnik odabire opciju "Spremi promjene"
							\item Baza podataka se ažurira
					
						\end{packed_enum}

						\item  \textbf{Opis mogućih odstupanja:}
						
						\item[] \begin{packed_item}
	
							\item[2.a]  Korisnik izvrši promjenu podataka, ali uneseni podaci nisu u ispravnom formatu
							\item[] \begin{packed_enum}
								
								\item Sustav ne dozvoljava spremanje izmjena dok podaci nisu u ispravnom formatu
								\item Sustav kraj podatka koji je neispravno zapisan prikaže obavijest kako podatak nije u ispravnom formatu te upućuje da korisnik ispravi isto
								
							\end{packed_enum}

							\item[2.b]  Korisnik izvrši promjenu podataka, ali ne odabere opciju "Spremi promjene"
							\item[] \begin{packed_enum}
								
								\item Sustav obavještava korisnika da nije spremio podatke prije izlaska iz prozora
								
							\end{packed_enum}
							
						\end{packed_item}
			
					\end{packed_item}

					\noindent \underbar{\textbf{UC6 - Predaja rada}}
					\begin{packed_item}
	
						\item \textbf{Glavni sudionik: } Sudionik konferencije
						\item  \textbf{Cilj:} Predati rad
						\item  \textbf{Sudionici:} Baza podataka
						\item  \textbf{Preduvjet:} Korisnik je prijavljen u sustav, rad još nije predan (.pdf nije učitan i pohranjen)
						\item  \textbf{Opis osnovnog tijeka:}
						
						\item[] \begin{packed_enum}

							\item Korisnik bira gumb "Moji radovi" iz glavnog izbornika	
							\item Na stranici "Moji radovi" korisnik odabire opciju "Predaj rad" kraj željenog rada
							\item Pojavljuju se već ispunjena polja s nazivom rada i skupom autora koje je korisnik pri registraciji naveo, korisnik odabire opciju "Učitaj rad".
							\item Korisnik učitava rad u .pdf formatu
							\item Korisnik odabire opciju "Zaključaj predaju"
							\item Rad se pohranjuje i baza podataka se ažurira

					
						\end{packed_enum}

						\item  \textbf{Opis mogućih odstupanja:}
						
						\item[] \begin{packed_item}
	
							\item[3.a]  Korisnik je učitao datoteku koja nije u *.pdf obliku
							\item[] \begin{packed_enum}
								
								\item Sustav javlja da je dozvoljeno učitavati isključivo .pdf dokumente i ne pohranjuje ono što je korisnik pokušao prenijeti
								
							\end{packed_enum}

							\item[3.b]  Korisnik prenese datoteku, ali ne odabere opciju „Zaključaj predaju“
							\item[] \begin{packed_enum}
								
								\item Sustav javlja korisniku kako nije pohranio rad prije izlaska iz prozora
								
							\end{packed_enum}
							
						\end{packed_item}
			
					\end{packed_item}

					\noindent \underbar{\textbf{UC7 - Zahtjev za predaju dodatnog rada}}
					\begin{packed_item}
	
						\item \textbf{Glavni sudionik: } Sudionik konferencije
						\item  \textbf{Cilj:} Predati zahtjev kako bi korisnik mogao predati još jedan rad
						\item  \textbf{Sudionici:} Baza podataka
						\item  \textbf{Preduvjet:} Korisnik je prijavljen u sustav, prethodno je predao minimalno jedan rad
						\item  \textbf{Opis osnovnog tijeka:}
						
						\item[] \begin{packed_enum}
							\item Korisnik bira gumb "Moji radovi" iz glavnog izbornika
							\item Na stranici "Moji radovi" korisnik odabire opciju "Želim predati još jedan rad"
							\item Korisnik ispunjava formu za predaju zahtjeva
							\item Korisnik odabire opciju "Predaj zahtjev"
							\item Korisnik dobiva odgovor na zahtjev putem e-maila, a nakon što se prijavi u sustav (ukoliko mu je zahtjev bio odobren) na stranici "Moji radovi" vidi još jedan rad na popisu, odnosno naziv onog rada za koji je predao zahtjev

					
						\end{packed_enum}
			
					\end{packed_item}

					\noindent \underbar{\textbf{UC8 - Odobravanje registracije recenzenta}}
					\begin{packed_item}
	
						\item \textbf{Glavni sudionik: } Predsjedavajući konferencije
						\item  \textbf{Cilj:} Odobriti registraciju recenzenta i poslati mu podatke za prijavu
						\item  \textbf{Sudionici:} Baza podataka
						\item  \textbf{Preduvjet:} Korisnik je prijavljen u sustav
						\item  \textbf{Opis osnovnog tijeka:}
						
						\item[] \begin{packed_enum}
	
							\item Korisnik odabire gumb "Korisnici i radovi" iz izbornika, nakon što mu se učita stranica odabire "Popis recenzenata koje čekaju odobrenje" te mu se učita stranica s popisom svih registracija koje čekaju odobrenje
							\item Korisnik (ukoliko želi) unosi ograničenja u filtar (po državi, po sekciji) te mu se sukladno prikazuje popis recenzenata koji odgovaraju rezultatima filtra
							\item Korisnik odabire recenzenta s popisa
							\item Korisniku se prikazuju podaci koje je recenzent unio pri registraciji
							\item Korisnik odabire opciju "Potvrdi prijavu i pošalji potvrdu"
							\item Šalje se poruka o uspješnoj registraciji (koju je potrebno potvrditi "klikom" na dobiveni link) i podatci za prijavu putem e-maila odabranom recenzentu
							\item Ažurira se baza podataka

					
						\end{packed_enum}
			
					\end{packed_item}

					\noindent \underbar{\textbf{UC9 - Odbijanje registracije recenzenta}}
					\begin{packed_item}
	
						\item \textbf{Glavni sudionik: } Predsjedavajući konferencije
						\item  \textbf{Cilj:} Odbiti registraciju recenzenta i izbrisati ga iz baze podataka
						\item  \textbf{Sudionici:} Baza podataka
						\item  \textbf{Preduvjet:} Korisnik je prijavljen u sustav
						\item  \textbf{Opis osnovnog tijeka:}
						
						\item[] \begin{packed_enum}

							\item Korisnik odabire gumb "Korisnici i radovi" iz izbornika, nakon što mu se učita stranica odabire "Popis recenzenata koje čekaju odobrenje" te mu se učita stranica s popisom svih registracija koje čekaju odobrenje
							\item Korisnik (ukoliko želi) unosi ograničenja u filtar (po državi, po sekciji) te mu se sukladno prikazuje popis sudionika koji odgovaraju rezultatima filtra
							\item Korisnik odabire registraciju s popisa
							\item Korisniku se prikazuju podaci koje je recenzent unio pri registraciji
							\item Korisnik odabire opciju "Odbij prijavu"
							\item Otvara se polje za unos obrazloženja odbijanja prijave
							\item Korisnik navodi razlog/e i odabire opciju "Pošalji odbijenicu"
							\item Šalje se personalizirana poruka putem e-maila odabranom recenzentu
							\item Ažurira se baza podataka

					
						\end{packed_enum}
			
					\end{packed_item}


					\noindent \underbar{\textbf{UC10 - Pregled sudionika}}
					\begin{packed_item}
	
						\item \textbf{Glavni sudionik: } Administrator, predsjedavajući konferencije
						\item  \textbf{Cilj:} Pregledati popis uspješno registriranih sudionika
						\item  \textbf{Sudionici:} Baza podataka
						\item  \textbf{Preduvjet:} Korisnik je prijavljen u sustav
						\item  \textbf{Opis osnovnog tijeka:}
						
						\item[] \begin{packed_enum}
	
							\item Korisnik odabire gumb "Korisnici i radovi" iz izbornika, nakon što mu se učita stranica odabire "Popis registriranih sudionika" te mu se učita stranica s popisom svih sudionika
							\item Korisnik (ukoliko želi) unosi ograničenja u filtar (po državi) te mu se sukladno prikazuje popis sudionika koji odgovaraju rezultatima filtra
							\item Korisnik odabire sudionika s popisa
							\item Korisniku se prikazuju informacije o odabranom sudioniku

					
						\end{packed_enum}
			
					\end{packed_item}

					\noindent \underbar{\textbf{UC11 - Pregled recenzenata}}
					\begin{packed_item}
	
						\item \textbf{Glavni sudionik: } Administrator, predsjedavajući konferencije
						\item  \textbf{Cilj:} Pregledati popis registriranih i odobrenih recezenata
						\item  \textbf{Sudionici:} Baza podataka
						\item  \textbf{Preduvjet:} Korisnik je prijavljen u sustav
						\item  \textbf{Opis osnovnog tijeka:}
						
						\item[] \begin{packed_enum}

							\item Korisnik odabire gumb "Korisnici i radovi" iz izbornika, nakon što mu se učita stranica odabire "Popis odobrenih recenzenata" te mu se učita stranica s popisom svih recenzenata
							\item Korisnik (ukoliko želi) unosi ograničenja u filtar (po državi) te mu se sukladno prikazuje popis recenzenata koji odgovaraju rezultatima filtra
							\item Korisnik odabire recenzenta s popisa
							\item Korisniku se prikazuju informacije o odabranom recenzentu

					
						\end{packed_enum}
			
					\end{packed_item}

					\noindent \underbar{\textbf{UC12 - Dodavanje novog administratora}}
					\begin{packed_item}
	
						\item \textbf{Glavni sudionik: } Administrator
						\item  \textbf{Cilj:} Dodati novog administratora u bazu podataka
						\item  \textbf{Sudionici:} Baza podataka
						\item  \textbf{Preduvjet:} Korisnik je prijavljen u sustav
						\item  \textbf{Opis osnovnog tijeka:}
						
						\item[] \begin{packed_enum}
	
							\item Korisnik bira gumb "Administracijsko sučelje" iz glavnog izbornika
							\item Korisnik u administracijskom sučelju bira opciju "Dodaj novog administratora"
							\item Korisnik unosi osobne podatke (ime, prezime, e-mail) administratora kojeg želi dodati
							\item Korisnik odabire opciju "Dodaj administratora u sustav"
							\item Na uneseni e-mail šalju se podatci za prijavu u sustav (korisničko ime i lozinka).
							\item Ažurira se baza podataka
							

					
						\end{packed_enum}

						\item  \textbf{Opis mogućih odstupanja:}
						
						\item[] \begin{packed_item}
							
							\item[3.a]  Korisnik unese podatke u neispravnom obliku ili izostavi neki od podataka.
							\item[] \begin{packed_enum}
								
								\item Sustav ne dozvoljava spremanje izmjena dok svi podaci nisu uneseni u ispravnom obliku.
								\item Sustav naznačuje koji podatak nedostaje ili je u neispravnom obliku.
								
							\end{packed_enum}
	
							\item[4.a]  Korisnik unese podatke o administratoru, ali ne odabere opciju "Dodaj administratora u sustav"
							\item[] \begin{packed_enum}
								
								\item Sustav javlja korisniku kako novi administrator nije dodan u sustav prije izlaska iz prozora
								
							\end{packed_enum}
						
							
						\end{packed_item}
			
					\end{packed_item}
				
				
					\noindent \underbar{\textbf{UC13 - Recenziranje rada}}
					\begin{packed_item}
						
						\item \textbf{Glavni sudionik:} Recenzent
						\item \textbf{Cilj:} Pregledati i ocijeniti rad
						\item \textbf{Sudionici:} Baza podataka
						\item \textbf{Preduvjet:} Korisnik je prijavljen u sustav
						\item \textbf{Opis osnovnog tijeka:} 
						
						\item[] \begin{packed_enum}
							\item Korisnik bira gumb "Radovi" iz glavnog izbornika te mu se učitava stranica sa radovima dostupnima za recenziju
							\item Korisnik odabire jedan od ponuđenih radova za recenziju
							\item  Korisnik na stranici vidi prikaz dokumenta kojeg recenzira (uz opciju "Preuzmi rad" koja pohranjuje .pdf lokalno)
							\item Korisnik bira jednu od 4 moguće ocjene te ovisno o odabranoj ocjeni u polje za obrazloženje navodi razlog svog odabira
							\item Korisnik odabire opciju "Zaključaj recenziju i obavijesti autora/e rada"
							\item Recenzija se pohranjuje i šalje autoru/ima (onom koji je pri registraciji označen kao "osoba za kontakt") rada na e-mail i baza podataka se ažurira
						\end{packed_enum}
					
						\item \textbf{Opis mogućih odstupanja:}
						
						\item[] \begin{packed_enum}
							\item[4.a] Korisnik nije odabrao ocjenu ili upisao obrazloženje
							\item[] \begin{packed_enum}
								\item[1.] Sustav javlja da se ocjena mora obrazloženje i da komentar mora biti upisano
							\end{packed_enum}
							\item[4.b] Korisnik je odabrao ocjenu i upisao obrazloženje, ali nije odabrao opciju "Pošalji recenziju"
							\item[] \begin{packed_enum}
								\item[1.] Sustav javlja korisniku kako nije pohranio recenziju prije izlaska iz prozora
							\end{packed_enum}
						\end{packed_enum}
					
					\end{packed_item}
				
					\noindent \underbar{\textbf{UC14 - Izmjena rada po zahtjevu recenzenta}}
					\begin{packed_item}
						\item \textbf{Glavni sudionik:} Sudionik
						\item \textbf{Cilj:} Izmijeniti i učitati ranije učitan rad
						\item \textbf{Sudionici:} Baza podataka
						\item \textbf{Preduvjet:} Korisnik je prijavljen u sustav, recenzent dao ocjenu za koju je rad potrebno izmijeniti
						
						\item \textbf{Opis osnovnog tijeka:} 
						\item[] \begin{packed_enum}
							\item Korisnik bira gumb "Moji radovi" iz glavnog izbornika te se učitava stranica s radovima koje je korisnik predao
							\item Korisnik odabire rad koji želi izmijeniti
							\item Korisnik učitava izmijenjeni rad u .pdf formatu
							\item Korisnik odabire opciju "Zaključi predaju"
							\item Ako je rad ocijenjen da su potrebne veće izmjene, automatski se šalje e-mail recenzentu rada o obavljenoj izmjeni
							\item Rad se pohranjuje i baza podataka se ažurira
						\end{packed_enum}
					
						\item \textbf{Opis mogućih odstupanja:}
						\item[] \begin{packed_enum}
							\item[2.a] Korisnik je učitao datoteku koja nije u .pdf formatu
							\item[] \begin{packed_enum}
								\item[1.] Sustav javlja da je dozvoljeno učitati isključivo .pdf dokumente i ne pohranjuje ono što je korisnik pokušao prenijeti
							\end{packed_enum}
						
							\item[2.b] Korisnik prenese datoteku, ali ne odabere opciju "Zaključi predaju"
							\item[] \begin{packed_enum}
								\item[1.] Sustav javlja korisniku kako nije pohranio rad prije izlaska iz prozora
							\end{packed_enum}
							
						\end{packed_enum}
					\end{packed_item}
					
					\noindent \underbar{\textbf{UC15 - Pregled radova}}
					\begin{packed_item}
	
						\item \textbf{Glavni sudionik: } Administrator, Predsjedavajući konferencije
						\item  \textbf{Cilj:} Pregledati popis predanih radova
						\item  \textbf{Sudionici:} Baza podataka
						\item  \textbf{Preduvjet:} Korisnik je prijavljen u sustav
						\item  \textbf{Opis osnovnog tijeka:}
						
						\item[] \begin{packed_enum}
	
							\item Korisnik odabire gumb "Korisnici i radovi" iz izbornika, nakon što mu se učita stranica odabire "Predani radovi" te mu se učita stranica s popisom svih radova
							\item Korisnik (ukoliko želi) unosi ograničenja u filtar (po sekciji, po državi) te mu se sukladno prikazuje popis radova koji odgovaraju rezultatima filtra
							\item Korisnik odabire rad s popisa
							\item Korisniku se prikazuju informacije o odabranom radu
							\item Korisnik, ukoliko on to želi, odabire opciju "Preuzmi rad" te preuzima rad lokalno sebi na računalo

					
						\end{packed_enum}
			
					\end{packed_item}
				
					\noindent \underbar{\textbf{UC16 - Učini radove javno dostupnima}}
					
					\begin{packed_item}
						
						\item \textbf{Glavni sudionik: } Administrator
						\item  \textbf{Cilj:} Učiniti radove javno dostupnima u slučaju kad se ukaže potreba za tim
						\item  \textbf{Sudionici:} Baza podataka
						\item  \textbf{Preduvjet:} Korisnik je prijavljen u sustav, postoji minimalno 1 rad koji nije javno dostupan
						\item  \textbf{Opis osnovnog tijeka:}
						
						\item[] \begin{packed_enum}
							\item Korisnik odabire gumb "Korisnici i radovi" iz izbornika, nakon što mu se učita stranica odabire "Predani radovi" te mu se učita stranica s popisom svih radova
							\item Korisnik (ukoliko želi) unosi parametre u filtar (po državi, po sekciji) te mu se nakon toga prikazuje popis radova s rezultatima filtra
							\item	 Korisnik označi željene radove s popisa koji još nemaju javni pristup, ili odabere opciju "Odaberi sve" ukoliko sve radove treba učiniti javno dostupnima
							\item Korisnik odabire opciju "Učini radove javno dostupnima"
							\item Korisniku se otvara prozor s porukom "Jeste li sigurni da želite objaviti odabrane radove?"
							\item Korisniku odabire opciju "Da"
							\item Korisnika sustav obavještava da su radovi uspješno objavljeni
							\item Baza podataka se ažurira
							
						\end{packed_enum}
					
						\item \textbf{Opis mogućih odstupanja:}
						\item[] \begin{packed_enum}
							\item[3.a] Korisnik odabere neke radove koji već jesu javno dostupni
							\item[] \begin{packed_enum}
								\item[1.] Sustav javlja korisniku da odznači radove koji već jesu javno dostupni, do tada mu nije dostupna opcija "Učini radove javno dostupnima"
							\end{packed_enum}
							\item[6.a] Korisnik odabire opciju "Ne"
							\item[] \begin{packed_enum}
								\item[1.] Korisnik je vraćen na stranicu popisa predanih radova
							\end{packed_enum} 
						\end{packed_enum}
						
					\end{packed_item}

					\noindent \underbar{\textbf{UC17 - Slanje obavijesti sudionicima i recenzentima}}
					\begin{packed_item}
						\item \textbf{Glavni sudionik:} Administrator, predsjedavajući konferencija
						\item \textbf{Cilj:} Obavijestiti sve sudionike i recenzente o promjeni ili novosti vezane za konferenciju
						\item \textbf{Sudionici:} Baza podataka
						\item \textbf{Preduvjet:} Korisnik je prijavljen u sustav
						
						\item \textbf{Opis osnovnog tijeka:} 
						\item[] \begin{packed_enum}
							\item Korisnik bira gumb "Administracijsko sučelje" iz glavnog izbornika
							\item Korisnik na administracijskom sučelju bira opciju "Pošalji obavijest korisnicima"
							\item Pojavljuje se popis svih sudionika i recenzenata i tražilica pomoću koje se taj popis može pretraživati
							\item Korisnik odabire 1 ili više primatelja obavijesti, a također ima i opciju "Označi sve"
							\item Pojavljuje se polje za unos teksta obavijesti te korisnik piše tekst obavijesti
							\item Korisnik bira opciju "Pošalji"
							\item Svi primatelji primaju obavijest putem e-maila s kojim su se registrirali
						\end{packed_enum}
					
						\item \textbf{Opis mogućih odstupanja:}
						\item[] \begin{packed_enum}
							
		

							\item[6.a] Korisnik unese tekst obavijesti, ali ne odabere opciju "Pošalji"
							\item[] \begin{packed_enum}
								\item[1.] Sustav javlja korisniku kako obavijest nije poslana korisnicima prije izlaska iz prozora
							\end{packed_enum}
							item[6.b] Korisnik ne odabere ni jednog primatelja obavijesti, a odabere opciju "Pošalji".
							\item[] \begin{packed_enum}
								\item[1.] Sustav javlja korisniku kako mora odabrati barem jednog primatelja.
							\end{packed_enum}
							item[6.c] Korisnik ne upiše ništa u polje za unos teksta, a odabere opciju "Pošalji".
							\item[] \begin{packed_enum}
								\item[1.] Sustav upozorava korisnika da nije upisao tekst obavijesti.
							\end{packed_enum}
							
						\end{packed_enum}
					\end{packed_item}


					\noindent \underbar{\textbf{UC18 - Definiranje upitnika za registraciju}}
					\begin{packed_item}
						\item \textbf{Glavni sudionik:} Administrator
						\item \textbf{Cilj:} Urediti upitnik za registraciju
						\item \textbf{Sudionici:} Baza podataka
						\item \textbf{Preduvjet:} Korisnik je prijavljen u sustav
						
						\item \textbf{Opis osnovnog tijeka:} 
						\item[] \begin{packed_enum}
							\item Korisnik bira gumb "Administracijsko sučelje" iz glavnog izbornika
							\item Korisnik na administracijskom sučelju bira opciju "Definiraj upitnik"
							\item Korisnik "kvačicama" označava polja za pitanja koja želi uključiti u upitnik, a odznačava ona koje želi ukloniti (dio polja je predefiniran i ne može ih odznačiti)
							\item Korisnik bira opciju "Zaključaj"
							\item Ažurira se baza podataka i upitnik
						\end{packed_enum}
					
						\item \textbf{Opis mogućih odstupanja:}
						\item[] \begin{packed_enum}

							\item[4.a] Korisnik unese promjene, ali ne odabere opciju "Zaključaj"
							\item[] \begin{packed_enum}
								\item[1.] Sustav javlja korisniku kako promjene nisu pohranjene prije izlaska iz prozora
							\end{packed_enum}
							
						\end{packed_enum}
					\end{packed_item}


					\noindent \underbar{\textbf{UC19 - Upravljanje sadržajem naslovnice}}
					\begin{packed_item}
						\item \textbf{Glavni sudionik:} Administrator, predsjedavajući konferencije
						\item \textbf{Cilj:} Ažurirati naslovnicu
						\item \textbf{Sudionici:} Baza podataka
						\item \textbf{Preduvjet:} Korisnik je prijavljen u sustav
						
						\item \textbf{Opis osnovnog tijeka:} 
						\item[] \begin{packed_enum}
							\item Korisnik bira gumb "Naslovnica" iz izbornika pri čemu pristupa sučelju za uređivanje naslovnice
							\item Korisnik na željenom odsječku bira opciju „Uredi“
							\item Korisnik unosi promjene (uređuje tekst)
							\item Korisnik bira opciju „Spremi“
						\end{packed_enum}
					
						\item \textbf{Opis mogućih odstupanja:}
						\item[] \begin{packed_enum}

							\item[3.a] Korisnik unese promjene, ali ne odabere opciju "Spremi"
							\item[] \begin{packed_enum}
								\item[1.] Sustav javlja korisniku kako promjene nisu spremljene prije izlaska iz prozora
							\end{packed_enum}
							
						\end{packed_enum}
					\end{packed_item}

					
					\noindent \underbar{\textbf{UC20 - Uređivanje podataka o sudioniku}}
					\begin{packed_item}
						\item \textbf{Glavni sudionik:} Administrator, Predsjedavajući konferencije
						\item \textbf{Cilj:} Uređivanje osobnih podataka o sudioniku, ispravljanje grešaka
						\item \textbf{Sudionici:} Baza podataka
						\item \textbf{Preduvjet:} Korisnik je prijavljen u sustav
						
						\item \textbf{Opis osnovnog tijeka:} 
						\item[] \begin{packed_enum}
							\item Korisnik odabire gumb "Korisnici i radovi" iz izbornika, nakon što mu se učita stranica odabire "Registrirani sudionici" te mu se učita stranica s popisom svih sudionika
							\item Korisnik (ukoliko želi) unosi parametre u filtar (po državi) te mu se nakon toga prikazuje popis sudionika s rezultatima filtra
							\item Korisnik odabere željenog sudionika te bira opciju "Uredi podatke o korisniku"
							\item Korisnik mijenja željene podatke
							\item Korisnik bira opciju "Spremi promjene"
						\end{packed_enum}
					
						\item \textbf{Opis mogućih odstupanja:}
						\item[] \begin{packed_enum}

							\item[5.a] Korisnik unese promjene, ali ne odabere opciju "Spremi promjene"
							\item[] \begin{packed_enum}
								\item[1.] Sustav javlja korisniku kako promjene nisu spremljene prije izlaska iz prozora
							\end{packed_enum}
						\end{packed_enum}
					\end{packed_item}

					\noindent \underbar{\textbf{UC21 - Uređivanje podataka o recenzentu}}
					\begin{packed_item}
						\item \textbf{Glavni sudionik:} Administrator, Predsjedavajući konferencije
						\item \textbf{Cilj:} Uređivanje osobnih podataka o recenzentu, ispravljanje grešaka
						\item \textbf{Sudionici:} Baza podataka
						\item \textbf{Preduvjet:} Korisnik je prijavljen u sustav
						
						\item \textbf{Opis osnovnog tijeka:} 
						\item[] \begin{packed_enum}
							\item Korisnik odabire gumb "Korisnici i radovi" iz izbornika, nakon što mu se učita stranica odabire "Registrirani recenzenti" te mu se učita stranica s popisom svih recenzenata
							\item Korisnik (ukoliko želi) unosi parametre u filtar (po državi) te mu se nakon toga prikazuje popis recenzenata s rezultatima filtra
							\item Korisnik odabere željenog sudionika te bira opciju "Uredi podatke o korisniku"
							\item Korisnik mijenja željene podatke
							\item Korisnik bira opciju "Spremi promjene"
						\end{packed_enum}
					
						\item \textbf{Opis mogućih odstupanja:}
						\item[] \begin{packed_enum}

							\item[4.a] Korisnik unese promjene, ali ne odabere opciju "Spremi promjene"
							\item[] \begin{packed_enum}
								\item[1.] Sustav javlja korisniku kako promjene nisu spremljene prije izlaska iz prozora
							\end{packed_enum}
						\end{packed_enum}
					\end{packed_item}

					\noindent \underbar{\textbf{UC22 - Ukloni javni pristup radova}}
					\begin{packed_item}
						\item \textbf{Glavni sudionik:} Administrator
						\item \textbf{Cilj:} Ukloniti javni pristup radovima
						\item \textbf{Sudionici:} Baza podataka
						\item \textbf{Preduvjet:} Korisnik je prijavljen u sustav, korisnik je prethodno dao javni pristup minimalno jednom radu
						
						\item \textbf{Opis osnovnog tijeka:} 
						\item[] \begin{packed_enum}
							\item Korisnik odabire gumb "Korisnici i radovi" iz izbornika,, nakon što mu se učita stranica odabire "Predani radovi" te mu se učita stranica s popisom svih radova
							\item Korisnik (ukoliko želi) unosi parametre u filtar (po državi, po sekciji) te mu se nakon toga prikazuje popis radova s rezultatima filtra
							\item Korisnik označi željene radove ili odabere opciju "Odaberi sve" ukoliko svi radovi s popisa imaju javni pristup
							\item Korisnik bira opciju "Ukloni javni pristup odabranim radovima"
							\item Korisniku se otvara prozor s porukom "Jeste li sigurni da želite ukloniti javni pristup ovim radovima?"
							\item Korisnik odabire opciju "Da"
							\item Korisnika sustav obavještava da su radovi uspješno uklonjeni s popisa javno dostupnih radova
							\item Baza podataka se ažurira
						\end{packed_enum}
					
						\item \textbf{Opis mogućih odstupanja:}
						\item[] \begin{packed_enum}
							\item[1.a] Korisnik odabere neke radove koji nisu javno dostupni
							\item[] \begin{packed_enum}
								\item[1.] Sustav javlja korisniku da odznači radove koji nisu javno dostupni, do tada mu nije dostupna opcija "Ukloniti javni pristup radovima"

							\end{packed_enum}
							\item[6.a] Korisnik odabire opciju "Ne"
							\item[] \begin{packed_enum}
								\item[1.] Korisnik je vraćen na stranicu popisa predanih radova
							\end{packed_enum} 
						\end{packed_enum}
					\end{packed_item}

					\noindent \underbar{\textbf{UC23 - Uređivanje podataka o radu}}
					\begin{packed_item}
						\item \textbf{Glavni sudionik:} Administrator
						\item \textbf{Cilj:} Uređivanje podataka o predanom radu
						\item \textbf{Sudionici:} Baza podataka
						\item \textbf{Preduvjet:} Korisnik je prijavljen u sustav
						
						\item \textbf{Opis osnovnog tijeka:} 
						\item[] \begin{packed_enum}
							\item Korisnik odabire gumb "Korisnici i radovi" iz izbornika, nakon što mu se učita stranica odabire "Predani radovi" te mu se učita stranica s popisom svih radova
							\item Korisnik bira opciju "Unesi promjene"
							\item Korisnik mijenja podatke o radu i/ili ažurira .pdf dokument
							\item Korisnik bira opciju "Spremi promjene"
						\end{packed_enum}
					
						\item \textbf{Opis mogućih odstupanja:}
						\item[] \begin{packed_enum}

							\item[3.a] Korisnik unese promjene, ali ne odabere opciju "Spremi promjene"
							\item[] \begin{packed_enum}
								\item[1.] Sustav javlja korisniku kako promjene nisu spremljene prije izlaska iz prozora
							\end{packed_enum}
							\item[3.b] Korisnik unese datoteku koja nije u .pdf formatu
							\item[] \begin{packed_enum}
								\item[1.] Sustav obavještava korisnika da nije dozvoljen unos datoteka koje nisu u .pdf formatu te u sustavu ostaje pohranjen stari .pdf
							\end{packed_enum}
						\end{packed_enum}
					\end{packed_item}
					
					\noindent \underbar{\textbf{UC25 - Zahtjev za recenziju dodatnog rada}}
					\begin{packed_item}
	
						\item \textbf{Glavni sudionik: } Recenzent
						\item  \textbf{Cilj:} Predati zahtjev kako bi korisnik mogao recenzirati još jedan rad
						\item  \textbf{Sudionici:} Baza podataka
						\item  \textbf{Preduvjet:} Korisnik je prijavljen u sustav, korisnik je prethodno recenzirao minimalno jedan rad
						\item  \textbf{Opis osnovnog tijeka:}
						
						\item[] \begin{packed_enum}
							\item Korisnik bira gumb "Moji radovi" iz glavnog izbornika te se učitava stranica s radovima koje je korisnik prijavio i/ili predao
							\item Korisnik odabire opciju "Želim recenzirati još jedan rad" na dnu stranice
							\item Korisnik ispunjava formu za predaju zahtjeva
							\item Korisnik odabire opciju "Predaj zahtjev"
							\item Korisnik dobiva odgovor na zahtjev putem e-maila

					
						\end{packed_enum}
					\end{packed_item}

					\noindent \underbar{\textbf{UC26 - Odobravanje zahtjeva za recenziju dodatnog rada}}
					\begin{packed_item}
	
						\item \textbf{Glavni sudionik: } Predsjedavajući konferencije
						\item  \textbf{Cilj:} Odobriti recenziju dodatnog rada
						\item  \textbf{Sudionici:} Baza podataka
						\item  \textbf{Preduvjet:} Korisnik je prijavljen u sustav
						\item  \textbf{Opis osnovnog tijeka:}
						
						\item[] \begin{packed_enum}
	
							\item Korisnik odabire gumb "Korisnici i radovi" iz izbornika, nakon što mu se učita stranica odabire opciju "Zahtjeva recenzenata koji čekaju odobrenje" te mu se učita stranica popisa zahtjeva
							\item Korisnik odabire zahtjev s popisa
							\item Korisniku se prikazuju podaci koje je recenzent unio pri predaji zahtjeva
							\item Korisnik odabire opciju "Odobri zahtjev i pošalji potvrdu"
							\item Šalje se automatska poruka putem mail-a odabranom recenzentu
							\item Ažurira se baza podataka
					
						\end{packed_enum}
			
					\end{packed_item}

					\noindent \underbar{\textbf{UC27 - Odbijanje zahtjeva za recenziranje dodatnog rada }}
					\begin{packed_item}
	
						\item \textbf{Glavni sudionik: } Predsjedavajući konferencije
						\item  \textbf{Cilj:} Odbiti recenziju dodatnog rada
						\item  \textbf{Sudionici:} Baza podataka
						\item  \textbf{Preduvjet:} Korisnik je prijavljen u sustav
						\item  \textbf{Opis osnovnog tijeka:}
						
						\item[] \begin{packed_enum}
	
							\item Korisnik odabire gumb "Korisnici i radovi" iz izbornika, nakon što mu se učita stranica odabire opciju "Zahtjeva recenzenata koji čekaju odobrenje" te mu se učita stranica popisa zahtjeva
							\item Korisnik odabire zahtjev s popisa
							\item Korisniku se prikazuju podaci koje je recenzent unio pri predaji zahtjeva
							\item Korisnik odabire opciju "Odbij prijavu"
							\item Otvara se polje za unos obrazloženja odbijanja prijave
							\item Korisnik navodi razlog/e i odabire opciju "Pošalji odbijenicu"
							\item Šalje se personalizirana poruka putem mail-a odabranom recenzentu
							\item Ažurira se baza podataka

					
						\end{packed_enum}
			
					\end{packed_item}

					\noindent \underbar{\textbf{UC28 - Pregled statistike prijava}}
					\begin{packed_item}
	
						\item \textbf{Glavni sudionik: } Administrator
						\item  \textbf{Cilj:} Pregledati informacije o prijavama na konferenciju
						\item  \textbf{Sudionici:} Baza podataka
						\item  \textbf{Preduvjet:} Korisnik je prijavljen u sustav
						\item  \textbf{Opis osnovnog tijeka:}
						
						\item[] \begin{packed_enum}
	
							\item Korisnik odabire gumb "Statistika prijava" iz izbornika, te se učitava stranica sa informacijama o prijavljenim radovima (ukupan broj radova te broj radova po sekcijama), popis država iz kojih su prijavljeni sudionici kao i broj sudionika po državi

						\end{packed_enum}
			
					\end{packed_item}
				
					\noindent \underbar{\textbf{UC29 - Praćenje podataka vezanih uz epidemiološku situaciju s COVID-19}}
					\begin{packed_item}
						
						\item \textbf{Glavni sudionik: } Administrator
						\item  \textbf{Cilj:} Pregledati informacije o incidenciji u pojednim državama iz kojih ima prijavljenih sudionika
						\item  \textbf{Sudionici:} Baza podataka
						\item  \textbf{Preduvjet:} Korisnik je prijavljen u sustav
						\item  \textbf{Opis osnovnog tijeka:}
						
						\item[] \begin{packed_enum}
							
							\item Korisnik odabire gumb "COVID-19" iz izbornika, te se učitava stranica sa popisom država iz kojih ima prijavljenih te javno dostupni podatci o incidenciji u tim državama 
							\item Korisnik pregledava podatke u organizacijske svrhe, odlučuje hoće li se konferencija održati online (u tom slučaju potrebno je o tome obavijestiti sve sudionike i učiniti radove javno dostupnima) ili će sudionici iz određenih država morati sudjelovati na daljinu  (u tom slučaju administrator treba poslati e-mail tim sudionicima i njihove radove učiniti javno dostupnima) 
							
							
						\end{packed_enum}
						
					\end{packed_item}
						\noindent \underbar{\textbf{UC30 - Pregled trenutno aktivnih registriranih korisnika}}
					\begin{packed_item}
						
						\item \textbf{Glavni sudionik: } Administrator
						\item  \textbf{Cilj:} Vidjeti tko su trenutno aktivni registrirani korisnici i koliko ih je
						\item  \textbf{Sudionici:} Baza podataka
						\item  \textbf{Preduvjet:} Korisnik je prijavljen u sustav
						\item  \textbf{Opis osnovnog tijeka:}
						
						\item[] \begin{packed_enum}
							
							\item Korisnik odabire gumb "Trenutno aktivni korisnici" koji je fiksno postavljen na nu svake stranice te mu se prikazuje broj trenutno aktivnih korisnika i njihova imena 
							
						\end{packed_enum}
						
					\end{packed_item}
					
					
					
				
					
				\subsubsection{Dijagrami obrazaca uporabe}
					
					\textit{Prikazati odnos aktora i obrazaca uporabe odgovarajućim UML dijagramom. Nije nužno nacrtati sve na jednom dijagramu. Modelirati po razinama apstrakcije i skupovima srodnih funkcionalnosti.}
				\eject		
				
			\subsection{Sekvencijski dijagrami}
				
				\textbf{\textit{dio 1. revizije}}\\
				
				\textit{Nacrtati sekvencijske dijagrame koji modeliraju najvažnije dijelove sustava (max. 4 dijagrama). Ukoliko postoji nedoumica oko odabira, razjasniti s asistentom. Uz svaki dijagram napisati detaljni opis dijagrama.}
				\eject
	
		\section{Ostali zahtjevi}
		
			\textbf{\textit{dio 1. revizije}}\\
		 
			 \textit{Nefunkcionalni zahtjevi i zahtjevi domene primjene dopunjuju funkcionalne zahtjeve. Oni opisuju \textbf{kako se sustav treba ponašati} i koja \textbf{ograničenja} treba poštivati (performanse, korisničko iskustvo, pouzdanost, standardi kvalitete, sigurnost...). Primjeri takvih zahtjeva u Vašem projektu mogu biti: podržani jezici korisničkog sučelja, vrijeme odziva, najveći mogući podržani broj korisnika, podržane web/mobilne platforme, razina zaštite (protokoli komunikacije, kriptiranje...)... Svaki takav zahtjev potrebno je navesti u jednoj ili dvije rečenice.}
			 
			 
			 
	